\chapter*{Extended abstract} % senza numerazione
\label{sommario}
\addcontentsline{toc}{chapter}{Extended abstract} % da aggiungere comunque all'indice


Mobile Ad-hoc NETworks are defined as a system of mobile nodes connected with each other via a wireless connection.
In this topic, Vehicular Ad-hoc NETworks are a particular case of MANETs where the nodes are vehicles equipped with communication devices. The idea of exchange information between static nodes and vehicular nodes has brought standardization bodies and automotive manufacturers to pay attention to its deployment and development.\vskip 1em
Applications in Vehicular communication networks can be classified into different categories. The two main categories are Vehicular-to-Vehicular communications (V2V) and Vehicular-to-Infrastructure  communications (V2I). In the first category the vehicles can set up an Independent Basic Service Set  without a controlling access point, exchanging information about traffic, speed, position and so on and so forth. Also, the vehicle can send information about problems in the car or problems with the driver to surrounding vehicles.\vskip 1em
In the second category, an Road side unit (RSU) can act as an access point with the Internet and forming a Basic Service Set with the vehicles in the network. Public authorities can use the data exchanged by the RSU with the vehicles in order to ease traffic flow and provide a real time response to congestion.\vskip 1em
IEEE 802.11p \cite{jiang2008ieee} is one of the recent amendments to the IEEE 802.11 standard that enables wireless access in vehicular environments (WAVE). IEEE 802.11p provides the basic radio standard for Dedicated Short Range Communications (DSRC). It is limited by the scope of IEEE 802.11, which is a MAC and PHY layer standard meant for single physical channel operation.The DSRC multi-channel setup and the operational concepts are taken care of by the upper layer IEEE 1609.
\vskip 1em
Another topic which gathers much interest is the performance measurement of realistic simulations of IVC protocols. An ideal scenario is to perform outdoor experiments, but setting the network and running an outdoor scenario can be dangerous and expensive. \vskip 1em
To overcome these problems, network simulation environments are commonly used to simulate the protocols before they are deployed in the real world. The quality of the results obtained by simulations is heavily influenced by the quality of the mobility models. There are different steps in the evolution of mobility models. At the beginning they relied on relatively simple models using random node movement. Later a crucial step was made using pre-recorded real-world mobility trace. These traces were obtained in outdoor experiments. While such a mobility model will arguably result in the most realistic vehicle movement in network simulations, its use is limited by this approaches’ inherent limitation to a small set of mobility parameters. Changing only one parameter, e.g. the density of vehicles, and keeping all other parameters unchanged is simply
infeasible in reasonably large scenarios. These problem can be overcome by generating these traces artificially. \\
In case where information about road conditions or warnings are transmitted over the network, a strictly interaction between the road traffic simulation and the network traffic is needed. In bidirectional simulations, two inter-dependent processes are running concurrently, the network simulator and the road traffic simulator. The two processes share information like position and speed of the simulated nodes, while other data are local.\newpage
A reason that made it really interesting to examine Vehicular Network is that they seem a credible opportunity to resolve traffic congestions, accidents, gives authorities real time information and so on and so forth. \\
This thesis aims to develop a Markov model based on the Gilbert model \cite{gilbert1960capacity}, a two state hidden Markov chain, in order to decrease the computational time used by the Veins framework \cite{sommer2011bid} to decode an incoming packet without losing accuracy. To do so, the new model is implemented and compared in term of time and accuracy with the actual model. \vskip 1em
\Cref{cha:VN} provides and introduction to the field of Vehicular Networks, their applications and the communication technologies used in this field.\\
\Cref{cha:frame} provides an introduction to the mobility models and their evolutions, the importance of network simulations to test new protocol before they are deployed in the real world. Furthermore, \Cref{cha:frame} introduces the frameworks used in this thesis. These are:
\begin{itemize}
    \item Veins: an open source Inter-Vehicular Communication simulation framework composed of an    event-based network simulator and a road traffic micro simulation model. The framework use OMNeT++ as network simulator and SUMO as road traffic simulator. Veins provides a comprehensive suite of IVC-specific models that can serve as a modular framework for simulating applications.
    \item SUMO: a microscopic road traffic simulators that uses the SK mobility model. SUMO allows high.performance simulations of huge networks with road consisting of multiple lanes, as well as intrajunction traffic on these roads. Vehicle types are freely configurable with each vehicle following statically assigned routes, dynamically generated routes, or driving according to a configured timetable.
    \item OMNeT++: a discrete event simulator for modeling communication networks, multi-processors and other distributed or parallel systems. Simulations are either run interactively, in a graphical environment, or are executed as command-line applications. The INET Framework extension used in Veins, provides provides a set of OMNeT++ modules that represent various layers of the Internet protocol suite, e.g., the TCP, UDP, IPv4, and ARP protocols.
\end{itemize}
\Cref{cha:PS} firstly provides an introduction to the mathematics background of the thesis, introducing the channel communications and the characteristics of a wireless signal. Then, it introduces the definition of Markov chain, the original model developed by Gilbert and a first possible solution to transform the model in a range-dependent Gilbert model. This because having the same parameters for each distance do not represent the characteristics of a communication link, which is strongly dependent on the distance. Then the network stack of the Veins framework is explained. As it explained in \Cref{sec:prob} the computation of the SNR and the SINR can be really expensive in large scenario with an high number of cars. The model proposed in this thesis aims to reduce the computation time by introducing an range-and-neighbors dependent Gilbert model. The changes made to the Veins framework are explained in \Cref{sec:implementation}.
\Cref{cha:eva} provides the evaluation of the new model. The scenario used is based on PLEXE \cite{segata2014plexe}, an extension of Veins which permits the realistic simulation of platooning protocols. Firstly, to estimate the parameters of the new model, three scenarios have been simulated, the first with 160 vehicles in the network, the second with 320 vehicles and the last with 640 vehicles. Each simulation has been repeated 10 times with different seeds for the generation of the random numbers. At the end only the simulations with 640 nodes has been used for estimating the parameter since it is the one that gives the best results.\\
As can be seen in \Cref{sec:estimation} a script has been used to extrapolate the data from the simulations and estimate the parameters of the new model.\\
The two models have been compared using two metrics, the efficiency and the accuracy of the results.
For the first metric the computation time of the two models have been computed.
As can be seen in \Cref{sec:gain}, the new model substantially decrease the computational time of the original model. \Cref{tab:duration} shows that there is a gain of the $76\%$ in execution time using the new model instead of the first, but this gain is useless if there is a low accuracy between the two models.
The second analysis analyze the accuracy between the two models. To measure the accuracy, the Packet Delivery Rate of the two models is computed and then the differences of packets exchanged by the two models has been measured using the sample mean and the standard deviation.
As can be seen in \Cref{sec:Resultaccuracy}, the result of the sample mean indicates that the original model correctly decoded in average $0.19\%$ packets more than the new model, while the result of the standard deviation indicates that there is only a deviation of $11.62\%$ from the sample mean.
In conclusion, it can be said that the new model is much faster than the original model without losing accuracy. The two models can be used together, the original model can be used in short simulations to estimate the parameters and then the new model can be used in long simulations saving up to days in computation times.\\
The contribution of the graduand for the work done in this thesis is determined by the implementation of the new model and the evaluation of it.
\newpage
